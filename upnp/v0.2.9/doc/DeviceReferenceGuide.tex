\documentclass{article}
\usepackage{docxx}
\begin{document}
\pagebreak
\sloppy
\cxxTitle{}
        {EBS UPNP DEVICE SDK}
        {}
        {}
        {}
\begin{cxxContents}
\cxxContentsEntry{1}{API Reference }{}
\cxxContentsEntry{2}{Internal Library Documentation}{}
\begin{cxxContents}
\cxxContentsEntry{2.1}{Discovery/SSDP}{}
\cxxContentsEntry{2.2}{Description}{}
\cxxContentsEntry{2.3}{Control/SOAP}{}
\cxxContentsEntry{2.4}{Eventing/GENA}{}
\end{cxxContents}
\cxxContentsEntry{3}{Examples }{}
\begin{cxxContents}
\cxxContentsEntry{3.1}{UPnP Device Initialization Example}{Setting up a UPnP Device }
\end{cxxContents}
\end{cxxContents}
\clearpage\pagebreak
\begin{cxxdoc}
\<center\>
Embedded UPnP Library version 1.0 \<br\>
\<br\>
\</center\>
\end{cxxdoc}
\begin{cxxentry}
{}
        {API Reference }
        {}
        {}
        {1}
\begin{cxxnames}
\cxxitem{int}
        {UPnP\_RuntimeInit}
        {(\ UPnPRuntime*\ rt,\ \ UPNP\_UINT8*\ serverAddr,\ \ UPNP\_UINT16\ serverPort,\ \ UPNP\_INT16\ ipType,\ \ UPNP\_CHAR*\ wwwRootDir,\ \ UPNP\_INT16\ maxConnections,\ \ UPNP\_INT16\ maxHelperThreads\ )}
        {Initialize a UPnPRuntime }
        {1.1}
\cxxitem{void}
        {UPnP\_RuntimeDestroy}
        {(\ UPnPRuntime*\ rt\ )}
        {Destroy a UPnPRuntime }
        {1.2}
\cxxitem{int}
        {UPnP\_AddVirtualFile}
        {(\ UPnPRuntime*\ rt,\ const\ UPNP\_CHAR*\ serverPath,\ const\ UPNP\_UINT8*\ data,\ UPNP\_INT32\ size,\ const\ UPNP\_CHAR*\ contentType\ )}
        {Create a virtual file on the HTTP server. }
        {1.3}
\cxxitem{int}
        {UPnP\_RemoveVirtualFile}
        {(\ UPnPRuntime*\ rt,\ const\ UPNP\_CHAR*\ serverPath\ )}
        {Remove a virtual file from the server }
        {1.4}
\cxxitem{int}
        {UPnP\_ProcessState}
        {(\ UPnPRuntime*\ rt,\ \ UPNP\_INT32\ msecTimeout\ )}
        {Process asynchronous operations in non-threaded mode. }
        {1.5}
\cxxitem{int}
        {UPnP\_StartDaemon}
        {(\ UPnPRuntime*\ rt\ )}
        {Start the UPnP Daemon thread. }
        {1.6}
\cxxitem{int}
        {UPnP\_StopDaemon}
        {(\ UPnPRuntime*\ rt,\ UPNP\_INT32\ secTimeout\ )}
        {Kill the UPnP Daemon thread. }
        {1.7}
\cxxitem{const\ UPNP\_CHAR*}
        {UPnP\_GetPropertyValueByName}
        {(\ IXML\_Document*\ propertySet,\ \ const\ UPNP\_CHAR*\ name\ )}
        {Get the value of a named property in a GENA notify message. }
        {1.8}
\cxxitem{const\ UPNP\_CHAR*}
        {UPnP\_GetPropertyNameByIndex}
        {(\ IXML\_Document*\ propertySet,\ \ int\ index\ )}
        {Get the name of the nth property. }
        {1.9}
\cxxitem{const\ UPNP\_CHAR*}
        {UPnP\_GetPropertyValueByIndex}
        {(\ IXML\_Document*\ propertySet,\ \ int\ index\ )}
        {Get the value of the nth property. }
        {1.10}
\cxxitem{int}
        {UPnP\_AddToPropertySet}
        {(\ IXML\_Document**\ doc,\ \ const\ UPNP\_CHAR*\ name,\ \ const\ UPNP\_CHAR*\ value\ )}
        {Add name and value pair to GENA notify message property set. }
        {1.11}
\cxxitem{int}
        {UPnP\_CreateActionResponse}
        {(\ UPnPActionRequest*\ request\ \ )}
        {Creates a SOAP action response message. }
        {1.12}
\cxxitem{IXML\_Document*}
        {UPnP\_CreateAction}
        {(\ const\ UPNP\_CHAR*\ serviceTypeURI,\ \ const\ UPNP\_CHAR*\ actionName\ )}
        {Create a SOAP action request. }
        {1.13}
\cxxitem{int}
        {UPnP\_SetActionArg}
        {(\ IXML\_Document*\ actionDoc,\ \ const\ UPNP\_CHAR*\ name,\ \ const\ UPNP\_CHAR*\ value\ )}
        {Set an argument for a SOAP action response/request }
        {1.14}
\cxxitem{int}
        {UPnP\_DeviceInit}
        {(\ UPnPDeviceRuntime*\ deviceRuntime,\ \ UPnPRuntime*\ rt)}
        {Initialize a UPnPDeviceRuntime }
        {1.15}
\cxxitem{void}
        {UPnP\_DeviceFinish}
        {(\ UPnPDeviceRuntime*\ deviceRuntime)}
        {Destroy a UPnPDeviceRuntime }
        {1.16}
\cxxitem{int}
        {UPnP\_RegisterRootDevice}
        {(\ UPnPDeviceRuntime*\ deviceRuntime,\ \ const\ UPNP\_CHAR*\ descDocURL,\ \ IXML\_Document*\ description,\ \ UPNP\_BOOL\ autoAddr,\ \ UPnPDeviceCallback\ callback,\ \ void*\ userData,\ \ UPnPRootDeviceHandle*\ retHandle,\ \ UPNP\_BOOL\ deviceAdvertise)}
        {Configures the root device and its serives for UPnP }
        {1.17}
\cxxitem{int}
        {UPnP\_UnRegisterRootDevice}
        {(\ UPnPRootDeviceHandle\ rootDevice\ )}
        {Free root device from its server bindings }
        {1.18}
\cxxitem{int}
        {UPnP\_DeviceAdvertise}
        {(\ UPnPRootDeviceHandle\ rootDevice,\ \ UPNP\_INT32\ frequencySec,\ \ UPNP\_INT32\ remoteTimeoutSec)}
        {set up the device to send periodic SSDP announcements }
        {1.19}
\cxxitem{int}
        {UPnP\_DeviceNotify}
        {(\ UPnPDeviceRuntime*\ deviceRuntime,\ \ UPnPRootDeviceHandle\ rootDevice,\ \ const\ UPNP\_CHAR*\ deviceUDN,\ \ const\ UPNP\_CHAR*\ serviceId,\ \ IXML\_Document*\ propertySet)}
        {Sends an event notification message to all the subscribers of the service.}
        {1.20}
\cxxitem{int}
        {UPnP\_DeviceNotifyAsync}
        {(\ UPnPDeviceRuntime*\ deviceRuntime,\ \ UPnPRootDeviceHandle\ rootDevice,\ \ const\ UPNP\_CHAR*\ deviceUDN,\ \ const\ UPNP\_CHAR*\ serviceId,\ \ IXML\_Document*\ propertySet\ )}
        {Sends a non blocking event notification message to all the subscribers of the service.}
        {1.21}
\cxxitem{int}
        {UPnP\_AcceptSubscription}
        {(\ UPnPSubscriptionRequest*\ subReq,\ \ const\ GENA\_CHAR*\ subscriptionId,\ \ UPNP\_INT32\ timeoutSec,\ \ IXML\_Document*\ propertySet,\ \ UPNP\_INT32\ firstNotifyDelayMsec\ )}
        {Accept a new subscription request. }
        {1.22}
\cxxitem{int}
        {UPnP\_AcceptSubscriptionAsync}
        {(\ UPnPSubscriptionRequest*\ subReq,\ \ const\ GENA\_CHAR*\ subscriptionId,\ \ UPNP\_INT32\ timeoutSec,\ \ IXML\_Document*\ propertySet,\ \ UPNP\_INT32\ firstNotifyDelayMsec\ )}
        {Send Subscription Accept asynchronously.}
        {1.23}
\cxxitem{const\ UPNP\_CHAR*}
        {UPnP\_GetRequestedDeviceName}
        {(\ void*\ eventStruct,\ \ enum\ e\_UPnPDeviceEventType\ eventType\ )}
        {Extracts Unique device name for the target device from control/subscription request}
        {1.24}
\cxxitem{const\ UPNP\_CHAR*}
        {UPnP\_GetRequestedServiceId}
        {(\ void*\ eventStruct,\ \ enum\ e\_UPnPDeviceEventType\ eventType\ )}
        {Extracts service identifier from a control/subscription request}
        {1.25}
\cxxitem{const\ UPNP\_CHAR*}
        {UPnP\_GetRequestedActionName}
        {(\ void*\ eventStruct,\ \ enum\ e\_UPnPDeviceEventType\ eventType\ )}
        {Extracts name of the target action from action/subscription request}
        {1.26}
\cxxitem{void}
        {UPnP\_SetActionErrorResponse}
        {(\ UPnPActionRequest*\ request,\ \ UPNP\_CHAR*\ description,\ \ UPNP\_INT32\ value\ \ )}
        {Sets error code and error description for a response to an action request}
        {1.27}
\cxxitem{const\ UPNP\_CHAR*}
        {UPnP\_GetArgValue}
        {(UPnPActionRequest*\ request,\ \ const\ UPNP\_CHAR*\ argName\ )}
        {Extracts the value of a given argument from an action.}
        {1.28}
\cxxitem{int}
        {UPnP\_SetActionResponseArg}
        {(UPnPActionRequest*\ request,\ \ const\ UPNP\_CHAR*\ name,\ \ const\ UPNP\_CHAR*\ value\ \ )}
        {Inserts name and value of an argument to an action response message}
        {1.29}
\end{cxxnames}
\begin{cxxfunction}
{int}
        {UPnP\_RuntimeInit}
        {(\ UPnPRuntime*\ rt,\ \ UPNP\_UINT8*\ serverAddr,\ \ UPNP\_UINT16\ serverPort,\ \ UPNP\_INT16\ ipType,\ \ UPNP\_CHAR*\ wwwRootDir,\ \ UPNP\_INT16\ maxConnections,\ \ UPNP\_INT16\ maxHelperThreads\ )}
        {Initialize a UPnPRuntime }
        {1.1}
\cxxParameter{
\begin{tabular}[t]{lp{0.5\textwidth}}
{\tt\strut rt} & pointer to uninitialized UPnPRuntime struct 
\\
{\tt\strut serverAddr} & ip address to bind HTTP server to (NULL for IP\_ADDR\_ANY) 
\\
{\tt\strut serverPort} & port to bind HTTP server to (0 for ANY\_PORT) 
\\
{\tt\strut ipType} & type of ip version used (ipv4 or ipv6), (RTP\_NET\_TYPE\_IPV4 for v4 and RTP\_NET\_TYPE\_IPV6 for v6) 
\\
{\tt\strut wwwRootDir} & HTTP root dir on local file system 
\\
{\tt\strut maxConnections} & the maximum limit on simultaneous HTTP server connections 
\\
{\tt\strut maxHelperThreads} & if UPNP\_MULTITHREAD is defined, the max number of helper threads to spawn 
\end{tabular}}
\cxxReturn{
\begin{tabular}[t]{lp{0.5\textwidth}}
{\tt\strut error} & code
\end{tabular}}
\begin{cxxdoc}

Initializes the given UPnPRuntime struct, and sets up an HTTP
server instance to receive control/event messages.  This function
must be called before any other function in the UPnP SDK.


\end{cxxdoc}
\end{cxxfunction}
\begin{cxxfunction}
{void}
        {UPnP\_RuntimeDestroy}
        {(\ UPnPRuntime*\ rt\ )}
        {Destroy a UPnPRuntime }
        {1.2}
\cxxParameter{
\begin{tabular}[t]{lp{0.5\textwidth}}
{\tt\strut rt} & pointer to UPnPRuntime struct 
\end{tabular}}
\cxxReturn{
\begin{tabular}[t]{lp{0.5\textwidth}}
{\tt\strut error} & code
\end{tabular}}
\begin{cxxdoc}

Must be called after all other UPnP SDK calls to clean up runtime
data for UPnP.


\end{cxxdoc}
\end{cxxfunction}
\begin{cxxfunction}
{int}
        {UPnP\_AddVirtualFile}
        {(\ UPnPRuntime*\ rt,\ const\ UPNP\_CHAR*\ serverPath,\ const\ UPNP\_UINT8*\ data,\ UPNP\_INT32\ size,\ const\ UPNP\_CHAR*\ contentType\ )}
        {Create a virtual file on the HTTP server. }
        {1.3}
\cxxReturn{
\begin{tabular}[t]{lp{0.5\textwidth}}
{\tt\strut error} & code
\end{tabular}}
\cxxSee{UPnP\_RemoveVirtualFile
\strut}
\begin{cxxdoc}

Makes the data buffer passed in available at the given path on the
HTTP server.


\end{cxxdoc}
\end{cxxfunction}
\begin{cxxfunction}
{int}
        {UPnP\_RemoveVirtualFile}
        {(\ UPnPRuntime*\ rt,\ const\ UPNP\_CHAR*\ serverPath\ )}
        {Remove a virtual file from the server }
        {1.4}
\cxxReturn{
\begin{tabular}[t]{lp{0.5\textwidth}}
{\tt\strut error} & code
\end{tabular}}
\begin{cxxdoc}

Must be called before UPnP\_RuntimeDestroy to remove any virtual files
added using UPnP\_AddVirtualFile.


\end{cxxdoc}
\end{cxxfunction}
\begin{cxxfunction}
{int}
        {UPnP\_ProcessState}
        {(\ UPnPRuntime*\ rt,\ \ UPNP\_INT32\ msecTimeout\ )}
        {Process asynchronous operations in non-threaded mode. }
        {1.5}
\cxxParameter{
\begin{tabular}[t]{lp{0.5\textwidth}}
{\tt\strut rt} & pointer to UPnPRuntime struct 
\\
{\tt\strut msecTimeout} & time in miliseconds for which the function blocks 
\end{tabular}}
\cxxReturn{
\begin{tabular}[t]{lp{0.5\textwidth}}
{\tt\strut error} & code
\end{tabular}}
\begin{cxxdoc}

This function blocks for at most msecTimeout milliseconds, processing
any asynchronous operations that may be in progress on either the
control point or device runtime attached to the given UPnPRuntime.

This function must be called in order to receive events if an
application is running with the UPnP SDK in single-threaded mode.


\end{cxxdoc}
\end{cxxfunction}
\begin{cxxfunction}
{int}
        {UPnP\_StartDaemon}
        {(\ UPnPRuntime*\ rt\ )}
        {Start the UPnP Daemon thread. }
        {1.6}
\cxxParameter{
\begin{tabular}[t]{lp{0.5\textwidth}}
{\tt\strut rt} & pointer to UPnPRuntime struct 
\end{tabular}}
\cxxReturn{
\begin{tabular}[t]{lp{0.5\textwidth}}
{\tt\strut error} & code
\end{tabular}}
\cxxSee{UPnP\_StopDaemon
\strut}
\begin{cxxdoc}

This function must be called in multithreaded mode to start the
UPnP daemon, which listens for requests/announcements on the network,
and sends any events to the attached control point/device runtime.


\end{cxxdoc}
\end{cxxfunction}
\begin{cxxfunction}
{int}
        {UPnP\_StopDaemon}
        {(\ UPnPRuntime*\ rt,\ UPNP\_INT32\ secTimeout\ )}
        {Kill the UPnP Daemon thread. }
        {1.7}
\cxxParameter{
\begin{tabular}[t]{lp{0.5\textwidth}}
{\tt\strut rt} & the device runtime to stop 
\\
{\tt\strut secTimeout} & time to wait for daemon to stop 
\end{tabular}}
\cxxReturn{
\begin{tabular}[t]{lp{0.5\textwidth}}
{\tt\strut error} & code
\end{tabular}}
\begin{cxxdoc}

This function stops the UPnP daemon from executing.  It will wait
for at most secTimeout seconds for all helper threads to terminate.
If this function returns negative error code, it means the timeout
expired without the successful termination of one or more helper
threads.  In this case, calling UPnP\_RuntimeDestroy may cause a fault
since there are still helper threads running that may try to access
the data structures pointed to by the UPnPRuntime.


\end{cxxdoc}
\end{cxxfunction}
\begin{cxxfunction}
{const\ UPNP\_CHAR*}
        {UPnP\_GetPropertyValueByName}
        {(\ IXML\_Document*\ propertySet,\ \ const\ UPNP\_CHAR*\ name\ )}
        {Get the value of a named property in a GENA notify message. }
        {1.8}
\cxxReturn{
\begin{tabular}[t]{lp{0.5\textwidth}}
{\tt\strut the} & value or NULL if the property was not found
\end{tabular}}
\begin{cxxdoc}

The string returned must not be modified in any way.  It is valid until
the IXML\_Document is deleted.


\end{cxxdoc}
\end{cxxfunction}
\begin{cxxfunction}
{const\ UPNP\_CHAR*}
        {UPnP\_GetPropertyNameByIndex}
        {(\ IXML\_Document*\ propertySet,\ \ int\ index\ )}
        {Get the name of the nth property. }
        {1.9}
\cxxParameter{
\begin{tabular}[t]{lp{0.5\textwidth}}
{\tt\strut propertySet} & address of xml property set 
\\
{\tt\strut index} & index in property for value 
\end{tabular}}
\cxxReturn{
\begin{tabular}[t]{lp{0.5\textwidth}}
{\tt\strut the} & value or NULL if the property was not found
\end{tabular}}
\begin{cxxdoc}

The string returned must not be modified in any way.  It is valid until
the IXML\_Document is deleted.


\end{cxxdoc}
\end{cxxfunction}
\begin{cxxfunction}
{const\ UPNP\_CHAR*}
        {UPnP\_GetPropertyValueByIndex}
        {(\ IXML\_Document*\ propertySet,\ \ int\ index\ )}
        {Get the value of the nth property. }
        {1.10}
\cxxParameter{
\begin{tabular}[t]{lp{0.5\textwidth}}
{\tt\strut propertySet} & address of xml property set 
\\
{\tt\strut index} & index in property for value 
\end{tabular}}
\cxxReturn{
\begin{tabular}[t]{lp{0.5\textwidth}}
{\tt\strut the} & value or NULL if the property was not found
\end{tabular}}
\begin{cxxdoc}

The string returned must not be modified in any way.  It is valid until
the IXML\_Document is deleted.


\end{cxxdoc}
\end{cxxfunction}
\begin{cxxfunction}
{int}
        {UPnP\_AddToPropertySet}
        {(\ IXML\_Document**\ doc,\ \ const\ UPNP\_CHAR*\ name,\ \ const\ UPNP\_CHAR*\ value\ )}
        {Add name and value pair to GENA notify message property set. }
        {1.11}
\cxxParameter{
\begin{tabular}[t]{lp{0.5\textwidth}}
{\tt\strut doc} & address of property set 
\\
{\tt\strut name} & pointer to name for new entry 
\\
{\tt\strut value} & address of value of for the new entry 
\end{tabular}}
\cxxReturn{
\begin{tabular}[t]{lp{0.5\textwidth}}
{\tt\strut error} & code
\end{tabular}}
\begin{cxxdoc}

Add a new name value pair entry to the property set


\end{cxxdoc}
\end{cxxfunction}
\begin{cxxfunction}
{int}
        {UPnP\_CreateActionResponse}
        {(\ UPnPActionRequest*\ request\ \ )}
        {Creates a SOAP action response message. }
        {1.12}
\cxxReturn{
\begin{tabular}[t]{lp{0.5\textwidth}}
{\tt\strut error} & code
\end{tabular}}
\begin{cxxdoc}

Creates a response message skeleton for the supplied SOAP action request


\end{cxxdoc}
\end{cxxfunction}
\begin{cxxfunction}
{IXML\_Document*}
        {UPnP\_CreateAction}
        {(\ const\ UPNP\_CHAR*\ serviceTypeURI,\ \ const\ UPNP\_CHAR*\ actionName\ )}
        {Create a SOAP action request. }
        {1.13}
\cxxParameter{
\begin{tabular}[t]{lp{0.5\textwidth}}
{\tt\strut serviceTypeURI} & string containing service type of the target service 
\\
{\tt\strut actionName} & name on action on the target service 
\end{tabular}}
\cxxReturn{
\begin{tabular}[t]{lp{0.5\textwidth}}
{\tt\strut pointer} & to newly created IXML\_Document, which can be passed intoUPnP\_SetActionArg to set the action arguments; NULL on error
\end{tabular}}
\begin{cxxdoc}

Creates an XML document which will hold the SOAP action request
message. This function returns the address of newly formed XML document.
After finishing the process of sending action request the application
must release this xml document.


\end{cxxdoc}
\end{cxxfunction}
\begin{cxxfunction}
{int}
        {UPnP\_SetActionArg}
        {(\ IXML\_Document*\ actionDoc,\ \ const\ UPNP\_CHAR*\ name,\ \ const\ UPNP\_CHAR*\ value\ )}
        {Set an argument for a SOAP action response/request }
        {1.14}
\cxxParameter{
\begin{tabular}[t]{lp{0.5\textwidth}}
{\tt\strut actionDoc} & pointer to action respose message 
\\
{\tt\strut name} & argument name *
\\
{\tt\strut value} & argument value *
\end{tabular}}
\cxxReturn{
\begin{tabular}[t]{lp{0.5\textwidth}}
{\tt\strut error} & code
\end{tabular}}
\begin{cxxdoc}

This function can be used on an IXML\_Document created by either
UPnP\_CreateActionResponse ($\rightarrow$ 1.13, {\em page }\pageref{cxx.1.13}) or UPnP\_CreateAction to set
either the input or output arguments for a SOAP action.


\end{cxxdoc}
\end{cxxfunction}
\begin{cxxfunction}
{int}
        {UPnP\_DeviceInit}
        {(\ UPnPDeviceRuntime*\ deviceRuntime,\ \ UPnPRuntime*\ rt)}
        {Initialize a UPnPDeviceRuntime }
        {1.15}
\cxxParameter{
\begin{tabular}[t]{lp{0.5\textwidth}}
{\tt\strut deviceRuntime} & pointer to the device runtime buffer
\\
{\tt\strut rt} & pointer to an initialized upnp runtimebuffer. 
\end{tabular}}
\cxxReturn{
\begin{tabular}[t]{lp{0.5\textwidth}}
{\tt\strut error} & code
\end{tabular}}
\cxxSee{UPnP\_DeviceFinish
\strut}
\begin{cxxdoc}
Initializes all device state data in a UPnPDeviceRuntime
struct (allocated by the calling application), and binds the
device to the specified UPnPRuntime.  The UPnPRuntime
must be initialized via UPnP\_RuntimeInit before this function
is called.  Only one device may be bound to a single
UPnPRuntime at once.  This function must be called before
all other device related functions.


\end{cxxdoc}
\end{cxxfunction}
\begin{cxxfunction}
{void}
        {UPnP\_DeviceFinish}
        {(\ UPnPDeviceRuntime*\ deviceRuntime)}
        {Destroy a UPnPDeviceRuntime }
        {1.16}
\cxxParameter{
\begin{tabular}[t]{lp{0.5\textwidth}}
{\tt\strut deviceRuntime} & address of runtime of device to destroy 
\end{tabular}}
\cxxReturn{
\begin{tabular}[t]{lp{0.5\textwidth}}
{\tt\strut error} & code
\end{tabular}}
\cxxSee{UPnP\_DeviceInit
\strut}
\begin{cxxdoc}
Cleans up all data associated with a UPnPDeviceRuntime structure.
Once this function has been called, it is safe to free the memory used
by the UPnPDeviceRuntime structure.


\end{cxxdoc}
\end{cxxfunction}
\begin{cxxfunction}
{int}
        {UPnP\_RegisterRootDevice}
        {(\ UPnPDeviceRuntime*\ deviceRuntime,\ \ const\ UPNP\_CHAR*\ descDocURL,\ \ IXML\_Document*\ description,\ \ UPNP\_BOOL\ autoAddr,\ \ UPnPDeviceCallback\ callback,\ \ void*\ userData,\ \ UPnPRootDeviceHandle*\ retHandle,\ \ UPNP\_BOOL\ deviceAdvertise)}
        {Configures the root device and its serives for UPnP }
        {1.17}
\cxxParameter{
\begin{tabular}[t]{lp{0.5\textwidth}}
{\tt\strut deviceRuntime} & device runtime information 
\\
{\tt\strut descDocURL} & relative url of devicedescription document 
\\
{\tt\strut description} & address of DOM representationof the device descriptiondocument 
\\
{\tt\strut autoAddr} & Select swtich for Auto IPif 1 - uses AutoIPif 0 - extracts address from thedevice description document 
\\
{\tt\strut callback} & pointer to the callback functionfor the device 
\\
{\tt\strut userData} & user data for callback 
\\
{\tt\strut retHandle} & handle to the current root device 
\\
{\tt\strut deviceAdvertise} & Switch to turn ON and OFF device advertisingIf 1 - device will be set up to send periodicSSDP announcements.If 0 - no ssdp announcements will be send
\end{tabular}}
\cxxReturn{
\begin{tabular}[t]{lp{0.5\textwidth}}
{\tt\strut error} & code
\end{tabular}}
\begin{cxxdoc}
Sets up the device to serve UPnP requests from the clients; set up
devcive for ssdp announcements if deviceAdvertise is turned on.


\end{cxxdoc}
\end{cxxfunction}
\begin{cxxfunction}
{int}
        {UPnP\_UnRegisterRootDevice}
        {(\ UPnPRootDeviceHandle\ rootDevice\ )}
        {Free root device from its server bindings }
        {1.18}
\cxxParameter{
\begin{tabular}[t]{lp{0.5\textwidth}}
{\tt\strut rootDevice} & Handle to root device 
\end{tabular}}
\cxxReturn{
\begin{tabular}[t]{lp{0.5\textwidth}}
{\tt\strut error} & code
\end{tabular}}
\begin{cxxdoc}
Unregisters the root device from the internal server, so that the future
UPnP requests will not be served for this root device.



\end{cxxdoc}
\end{cxxfunction}
\begin{cxxfunction}
{int}
        {UPnP\_DeviceAdvertise}
        {(\ UPnPRootDeviceHandle\ rootDevice,\ \ UPNP\_INT32\ frequencySec,\ \ UPNP\_INT32\ remoteTimeoutSec)}
        {set up the device to send periodic SSDP announcements }
        {1.19}
\cxxParameter{
\begin{tabular}[t]{lp{0.5\textwidth}}
{\tt\strut rootDevice} & handle to the device 
\\
{\tt\strut frequencySec} & interval in secondsbetween two announcements 
\\
{\tt\strut remoteTimeoutSec} & time in seconds for which the remoteclient will cache the information inthe announcement 
\end{tabular}}
\cxxReturn{
\begin{tabular}[t]{lp{0.5\textwidth}}
{\tt\strut error} & code
\end{tabular}}
\begin{cxxdoc}
The function prepares the device to send periodic announcements every
frequecySec seconds.

\end{cxxdoc}
\end{cxxfunction}
\begin{cxxfunction}
{int}
        {UPnP\_DeviceNotify}
        {(\ UPnPDeviceRuntime*\ deviceRuntime,\ \ UPnPRootDeviceHandle\ rootDevice,\ \ const\ UPNP\_CHAR*\ deviceUDN,\ \ const\ UPNP\_CHAR*\ serviceId,\ \ IXML\_Document*\ propertySet)}
        {Sends an event notification message to all the subscribers of the service.}
        {1.20}
\cxxParameter{
\begin{tabular}[t]{lp{0.5\textwidth}}
{\tt\strut deviceRuntime} & device runtime information 
\\
{\tt\strut rootDevice} & handle to the device 
\\
{\tt\strut deviceUDN} & unique device identifier (UUID in thedevice description document) forthe device 
\\
{\tt\strut serviceId} & unique service identifier (serviceIDin the device description document)for the service 
\\
{\tt\strut propertySet} & contains the evented variable andits value in XML format. 
\end{tabular}}
\cxxReturn{
\begin{tabular}[t]{lp{0.5\textwidth}}
{\tt\strut error} & code
\end{tabular}}
\begin{cxxdoc}
Sends an event notification message to all the subscribers of the service.


\end{cxxdoc}
\end{cxxfunction}
\begin{cxxfunction}
{int}
        {UPnP\_DeviceNotifyAsync}
        {(\ UPnPDeviceRuntime*\ deviceRuntime,\ \ UPnPRootDeviceHandle\ rootDevice,\ \ const\ UPNP\_CHAR*\ deviceUDN,\ \ const\ UPNP\_CHAR*\ serviceId,\ \ IXML\_Document*\ propertySet\ )}
        {Sends a non blocking event notification message to all the subscribers of the service.}
        {1.21}
\cxxParameter{
\begin{tabular}[t]{lp{0.5\textwidth}}
{\tt\strut deviceRuntime} & the runtime for the device
\\
{\tt\strut rootDevice} & the root device 
\\
{\tt\strut deviceUDN} & the UDN of the specific device 
\\
{\tt\strut serviceId} & the ID of the service notifying 
\\
{\tt\strut propertySet} & property set to send as new values 
\end{tabular}}
\cxxReturn{
\begin{tabular}[t]{lp{0.5\textwidth}}
{\tt\strut error} & code
\end{tabular}}
\begin{cxxdoc}
Sends a non blocking event notification message to all the subscribers of
the service.


\end{cxxdoc}
\end{cxxfunction}
\begin{cxxfunction}
{int}
        {UPnP\_AcceptSubscription}
        {(\ UPnPSubscriptionRequest*\ subReq,\ \ const\ GENA\_CHAR*\ subscriptionId,\ \ UPNP\_INT32\ timeoutSec,\ \ IXML\_Document*\ propertySet,\ \ UPNP\_INT32\ firstNotifyDelayMsec\ )}
        {Accept a new subscription request. }
        {1.22}
\cxxParameter{
\begin{tabular}[t]{lp{0.5\textwidth}}
{\tt\strut subReq} & address of structure containing subscriptionrequest information 
\\
{\tt\strut subscriptionId} & subscription identifier for the subscriber 
\\
{\tt\strut timeoutSec} & duration in seconds for which the subscriptionis valid 
\\
{\tt\strut propertySet} & pointer to response message in XML format.
\\
{\tt\strut firstNotifyDelayMsec} & delay in milliseconds before sending thefirst event notification to the new subscriber 
\end{tabular}}
\cxxReturn{
\begin{tabular}[t]{lp{0.5\textwidth}}
{\tt\strut error} & code
\end{tabular}}
\begin{cxxdoc}
This function adds a new subscriber device's internal subscriber's list,
generates a unique subscription Id for this subscriber, sets a duration
in seconds for this subscription to be valid and sends a subscription
response indicating success or failure to subscription request.


\end{cxxdoc}
\end{cxxfunction}
\begin{cxxfunction}
{int}
        {UPnP\_AcceptSubscriptionAsync}
        {(\ UPnPSubscriptionRequest*\ subReq,\ \ const\ GENA\_CHAR*\ subscriptionId,\ \ UPNP\_INT32\ timeoutSec,\ \ IXML\_Document*\ propertySet,\ \ UPNP\_INT32\ firstNotifyDelayMsec\ )}
        {Send Subscription Accept asynchronously.}
        {1.23}
\cxxParameter{
\begin{tabular}[t]{lp{0.5\textwidth}}
{\tt\strut subReq} & the request being accepted 
\\
{\tt\strut subscriptionId} & alternate subscription ID (optional) 
\\
{\tt\strut timeoutSec} & remote cache timeout value inseconds (optional) 
\\
{\tt\strut propertySet} & property set for initial notify 
\\
{\tt\strut firstNotifyDelayMsec} & delay in milliseconds before sending thefirst event notification to the new subscriber 
\end{tabular}}
\cxxReturn{
\begin{tabular}[t]{lp{0.5\textwidth}}
{\tt\strut error} & code
\end{tabular}}
\begin{cxxdoc}
Send Subscription Accept asynchronously. Optional parameters may be given
a value of zero to indicate use default.


\end{cxxdoc}
\end{cxxfunction}
\begin{cxxfunction}
{const\ UPNP\_CHAR*}
        {UPnP\_GetRequestedDeviceName}
        {(\ void*\ eventStruct,\ \ enum\ e\_UPnPDeviceEventType\ eventType\ )}
        {Extracts Unique device name for the target device from control/subscription request}
        {1.24}
\cxxParameter{
\begin{tabular}[t]{lp{0.5\textwidth}}
{\tt\strut eventStruct} & pointer to request structure 
\\
{\tt\strut eventType} & type of event 
\end{tabular}}
\cxxReturn{
\begin{tabular}[t]{lp{0.5\textwidth}}
{\tt\strut error} & code
\end{tabular}}
\begin{cxxdoc}
Extracts Unique device name for the target device from control/subscription request


\end{cxxdoc}
\end{cxxfunction}
\begin{cxxfunction}
{const\ UPNP\_CHAR*}
        {UPnP\_GetRequestedServiceId}
        {(\ void*\ eventStruct,\ \ enum\ e\_UPnPDeviceEventType\ eventType\ )}
        {Extracts service identifier from a control/subscription request}
        {1.25}
\cxxParameter{
\begin{tabular}[t]{lp{0.5\textwidth}}
{\tt\strut eventStruct} & pointer to request structure 
\\
{\tt\strut eventType} & type of event 
\end{tabular}}
\cxxReturn{
\begin{tabular}[t]{lp{0.5\textwidth}}
{\tt\strut error} & code
\end{tabular}}
\begin{cxxdoc}
Extracts service identifier from a control/subscription request


\end{cxxdoc}
\end{cxxfunction}
\begin{cxxfunction}
{const\ UPNP\_CHAR*}
        {UPnP\_GetRequestedActionName}
        {(\ void*\ eventStruct,\ \ enum\ e\_UPnPDeviceEventType\ eventType\ )}
        {Extracts name of the target action from action/subscription request}
        {1.26}
\cxxParameter{
\begin{tabular}[t]{lp{0.5\textwidth}}
{\tt\strut eventStruct} & pointer to request structure 
\\
{\tt\strut eventType} & type of event 
\end{tabular}}
\cxxReturn{
\begin{tabular}[t]{lp{0.5\textwidth}}
{\tt\strut pointer} & to action name is available
\end{tabular}}
\begin{cxxdoc}
Extracts name of the target action from action/subscription request


\end{cxxdoc}
\end{cxxfunction}
\begin{cxxfunction}
{void}
        {UPnP\_SetActionErrorResponse}
        {(\ UPnPActionRequest*\ request,\ \ UPNP\_CHAR*\ description,\ \ UPNP\_INT32\ value\ \ )}
        {Sets error code and error description for a response to an action request}
        {1.27}
\cxxReturn{
\begin{tabular}[t]{lp{0.5\textwidth}}
{\tt\strut none} & \end{tabular}}
\begin{cxxdoc}
Sets error code and error description for a response to an action request


\end{cxxdoc}
\end{cxxfunction}
\begin{cxxfunction}
{const\ UPNP\_CHAR*}
        {UPnP\_GetArgValue}
        {(UPnPActionRequest*\ request,\ \ const\ UPNP\_CHAR*\ argName\ )}
        {Extracts the value of a given argument from an action.}
        {1.28}
\cxxReturn{
\begin{tabular}[t]{lp{0.5\textwidth}}
{\tt\strut error} & code
\end{tabular}}
\begin{cxxdoc}
Extracts the value of a given argument from an action. Action information is
stored in form of IXML element.


\end{cxxdoc}
\end{cxxfunction}
\begin{cxxfunction}
{int}
        {UPnP\_SetActionResponseArg}
        {(UPnPActionRequest*\ request,\ \ const\ UPNP\_CHAR*\ name,\ \ const\ UPNP\_CHAR*\ value\ \ )}
        {Inserts name and value of an argument to an action response message}
        {1.29}
\cxxReturn{
\begin{tabular}[t]{lp{0.5\textwidth}}
{\tt\strut error} & code
\end{tabular}}
\begin{cxxdoc}
Inserts name and value of an argument to an action response message


\end{cxxdoc}
\end{cxxfunction}
\end{cxxentry}
\begin{cxxentry}
{}
        {Internal Library Documentation}
        {}
        {}
        {2}
\begin{cxxnames}
\cxxitem{}
        {Discovery/SSDP}
        {}
        {}
        {2.1}
\cxxitem{}
        {Description}
        {}
        {}
        {2.2}
\cxxitem{}
        {Control/SOAP}
        {}
        {}
        {2.3}
\cxxitem{}
        {Eventing/GENA}
        {}
        {}
        {2.4}
\end{cxxnames}
\begin{cxxdoc}
These functions are \<b\>not\</b\> at the API level and should not be called from
code outside the UPnP library. 
\end{cxxdoc}
\begin{cxxentry}
{}
        {Discovery/SSDP}
        {}
        {}
        {2.1}
\begin{cxxnames}
\cxxitem{SSDP\_INT32}
        {SSDP\_ServerInit}
        {(\ SSDPServerContext*\ ctx,\ \ SSDP\_UINT8*\ ipAddr,\ \ SSDP\_INT16\ ipType,\ \ const\ SSDP\_CHAR*\ serverId,\ \ SSDPCallback\ cb,\ \ void*\ cookie)}
        {SSDP server initialization routine. }
        {2.1.1}
\cxxitem{SSDP\_INT32}
        {SSDP\_ServerAddToSelectList}
        {(\ SSDPServerContext*\ ctx,\ RTP\_FD\_SET*\ readList,\ RTP\_FD\_SET*\ writeList,\ RTP\_FD\_SET*\ errList\ )}
        {}
        {2.1.2}
\cxxitem{SSDP\_BOOL}
        {SSDP\_ServerProcessState}
        {(\ SSDPServerContext*\ ctx,\ RTP\_FD\_SET*\ readList,\ RTP\_FD\_SET*\ writeList,\ RTP\_FD\_SET*\ errList\ )}
        {}
        {2.1.3}
\cxxitem{void}
        {SSDP\_ServerDestroy}
        {(\ SSDPServerContext*\ ctx\ pointer\ to\ SSDP\ context*\ /\ )}
        {}
        {2.1.4}
\cxxitem{SSDP\_INT32}
        {SSDP\_SendNotify}
        {(\ SSDPServerContext*\ ctx,\ \ const\ SSDP\_CHAR*\ notifyType,\ \ const\ SSDP\_CHAR*\ notifySubType,\ \ const\ SSDP\_CHAR*\ usn,\ \ const\ SSDP\_CHAR*\ location,\ \ SSDP\_UINT32*\ timeout)}
        {Send a SSDP notification for the device or service. }
        {2.1.5}
\cxxitem{SSDP\_INT32}
        {SSDP\_SendResponse}
        {(\ SSDPServerContext*\ ctx,\ \ SSDPPendingResponse*\ response)}
        {Deliver a responce to SSDP discovery request.}
        {2.1.6}
\cxxitem{SSDP\_INT32}
        {\_SSDP\_ProcessOneRequest}
        {(\ SSDPServerContext*\ ctx\ )}
        {Process an incoming SSDP discovery request. }
        {2.1.7}
\cxxitem{int}
        {SSDP\_ParseRequest}
        {(\ SSDPServerContext*\ ctx,\ \ SSDPServerRequest*\ ssdpRequest\ )}
        {Extract SSDP request. }
        {2.1.8}
\cxxitem{int}
        {SSDP\_McastRead}
        {(\ void*\ cookie,\ \ SSDP\_UINT8*\ buffer,\ \ SSDP\_INT32\ min,\ \ SSDP\_INT32\ max)}
        {Reads all messages posted to the multicast address}
        {2.1.9}
\cxxitem{int}
        {\_SSDP\_ReadMSearchHeader}
        {(\ void*\ request,\ \ HTTPSession*\ ptr,\ \ HTTPHeaderType\ type,\ \ const\ HTTP\_CHAR*\ name,\ \ const\ HTTP\_CHAR*\ value\ )}
        {Extracts MX and St headers from a SSDP request}
        {2.1.10}
\cxxitem{int}
        {\_SSDP\_ReadNotifyHeader}
        {(\ void*\ request,\ \ HTTPSession*\ ptr,\ \ HTTPHeaderType\ type,\ \ const\ HTTP\_CHAR*\ name,\ \ const\ HTTP\_CHAR*\ value\ )}
        {Extracts MX and St headers from a SSDP request}
        {2.1.11}
\cxxitem{SSDP\_INT32}
        {SSDP\_QueueSearchResponse}
        {(\ SSDPServerContext*\ ctx,\ \ SSDPSearch*\ search,\ \ const\ SSDP\_CHAR*\ targetLocation,\ \ const\ SSDP\_CHAR*\ targetURN,\ \ SSDP\_UINT32\ targetTimeoutSec)}
        {Queues a response to the response list based on its scheduled delivery time.}
        {2.1.12}
\cxxitem{SSDP\_INT32}
        {SSDP\_CheckPendingResponses}
        {(\ SSDPServerContext*\ ctx,\ \ SSDP\_UINT32\ currentTimeMsec)}
        {Delivers responses scheduled for delivery. }
        {2.1.13}
\cxxitem{void}
        {SSDP\_ProcessError}
        {(\ SSDP\_CHAR*\ errMsg)}
        {Process SSDP Errors}
        {2.1.14}
\cxxitem{SSDP\_UINT32}
        {SSDP\_RandMax}
        {(\ SSDP\_UINT32\ mxLimit)}
        {generates a random number between 0 and mxLimit}
        {2.1.15}
\cxxitem{int}
        {UPnP\_DeviceSSDPCallback}
        {(\ SSDPServerContext*\ ctx,\ \ SSDPServerRequest*\ serverRequest,\ \ void*\ cookie\ )}
        {SSDP callback for UPnP. }
        {2.1.16}
\cxxitem{int}
        {UPnP\_DeviceSendAllAlive}
        {(\ UPnPDeviceRuntime*\ runtime\ )}
        {Sends alive advertisements for everything under all the root devices. }
        {2.1.17}
\cxxitem{int}
        {UPnP\_DeviceSendRootDeviceAlive}
        {(\ UPnPRootDevice*\ rootDevice,\ \ int\ deep)}
        {Sends alive advertisements for a root device.}
        {2.1.18}
\cxxitem{int}
        {UPnP\_DeviceSendDeviceAlive}
        {(\ UPnPDevice*\ device)}
        {Sends alive notifications for a device}
        {2.1.19}
\cxxitem{int}
        {UPnP\_DeviceSendServiceAlive}
        {(\ UPnPService*\ service)}
        {Sends alive notifications for a service}
        {2.1.20}
\cxxitem{int}
        {\_UPnP\_DeviceSendDeviceAlive}
        {(\ UPnPDevice*\ device)}
        {Sends alive advertisements for a device.}
        {2.1.21}
\cxxitem{int}
        {\_UPnP\_DeviceSendServiceAlive}
        {(\ UPnPService*\ service)}
        {Sends alive advertisements for a service.}
        {2.1.22}
\cxxitem{int}
        {\_UPnP\_DeviceSendRootDeviceAlive}
        {(\ UPnPRootDevice*\ rootDevice,\ \ int\ deep)}
        {Sends alive advertisements for every thing under all root devices. }
        {2.1.23}
\cxxitem{int}
        {UPnP\_DeviceSendAllByeBye}
        {(\ UPnPDeviceRuntime*\ runtime)}
        {Sends bye-bye advertisements for each device and associated service for all the root devices.}
        {2.1.24}
\cxxitem{int}
        {UPnP\_DeviceSendRootDeviceByeBye}
        {(\ UPnPRootDevice*\ rootDevice,\ \ int\ deep)}
        {Sends bye-bye notifications for a root device}
        {2.1.25}
\cxxitem{int}
        {UPnP\_DeviceSendDeviceByeBye}
        {(\ UPnPDevice*\ device)}
        {Sends bye-bye notifications for the device}
        {2.1.26}
\cxxitem{int}
        {UPnP\_DeviceSendServiceByeBye}
        {(\ UPnPService*\ service)}
        {Sends bye-bye notifications for the service}
        {2.1.27}
\cxxitem{int}
        {\_UPnP\_DeviceSendRootDeviceByeBye}
        {(\ UPnPRootDevice*\ rootDevice,\ \ int\ deep)}
        {Sends bye-bye notifications for root device.}
        {2.1.28}
\cxxitem{int}
        {\_UPnP\_DeviceSendDeviceByeBye}
        {(\ UPnPDevice*\ device)}
        {Sends bye-bye notifications for the device}
        {2.1.29}
\cxxitem{int}
        {\_UPnP\_DeviceSendServiceByeBye}
        {(\ UPnPService*\ service)}
        {Sends bye-bye notifications a service.}
        {2.1.30}
\end{cxxnames}
\begin{cxxdoc}
These functions are \<b\>not\</b\> at the API level and should not be called from
code outside the UPnP library. 
\end{cxxdoc}
\begin{cxxfunction}
{SSDP\_INT32}
        {SSDP\_ServerInit}
        {(\ SSDPServerContext*\ ctx,\ \ SSDP\_UINT8*\ ipAddr,\ \ SSDP\_INT16\ ipType,\ \ const\ SSDP\_CHAR*\ serverId,\ \ SSDPCallback\ cb,\ \ void*\ cookie)}
        {SSDP server initialization routine. }
        {2.1.1}
\cxxParameter{
\begin{tabular}[t]{lp{0.5\textwidth}}
{\tt\strut ctx} & pointer to an uninitialized SSDP context structure 
\\
{\tt\strut ipAddr} & IP address of the host to bind the UDP socket to, if a NULL issupplied, the UDP socket is bound to the local IP address 
\\
{\tt\strut ipType} & ip version 4 or ipversion 6 
\\
{\tt\strut serverId} & String holding platform name 
\\
{\tt\strut cb} & SSDP Callback routine 
\\
{\tt\strut cookie} & cookie(runtime) to be stored in ssdp context to be passedlater to ssdp callback 
\end{tabular}}
\cxxReturn{
\begin{tabular}[t]{lp{0.5\textwidth}}
{\tt\strut error} & code
\end{tabular}}
\begin{cxxdoc}
This routine starts up SSDP services by creating a UDP socket, getting the ssdp
multicast membership for the socket and initlializing ssdp context.


\end{cxxdoc}
\end{cxxfunction}
\begin{cxxfunction}
{SSDP\_INT32}
        {SSDP\_ServerAddToSelectList}
        {(\ SSDPServerContext*\ ctx,\ RTP\_FD\_SET*\ readList,\ RTP\_FD\_SET*\ writeList,\ RTP\_FD\_SET*\ errList\ )}
        {}
        {2.1.2}
\end{cxxfunction}
\begin{cxxfunction}
{SSDP\_BOOL}
        {SSDP\_ServerProcessState}
        {(\ SSDPServerContext*\ ctx,\ RTP\_FD\_SET*\ readList,\ RTP\_FD\_SET*\ writeList,\ RTP\_FD\_SET*\ errList\ )}
        {}
        {2.1.3}
\end{cxxfunction}
\begin{cxxfunction}
{void}
        {SSDP\_ServerDestroy}
        {(\ SSDPServerContext*\ ctx\ pointer\ to\ SSDP\ context*\ /\ )}
        {}
        {2.1.4}
\end{cxxfunction}
\begin{cxxfunction}
{SSDP\_INT32}
        {SSDP\_SendNotify}
        {(\ SSDPServerContext*\ ctx,\ \ const\ SSDP\_CHAR*\ notifyType,\ \ const\ SSDP\_CHAR*\ notifySubType,\ \ const\ SSDP\_CHAR*\ usn,\ \ const\ SSDP\_CHAR*\ location,\ \ SSDP\_UINT32*\ timeout)}
        {Send a SSDP notification for the device or service. }
        {2.1.5}
\cxxParameter{
\begin{tabular}[t]{lp{0.5\textwidth}}
{\tt\strut ctx} & pointer to SSDP context 
\\
{\tt\strut notifyType} & the notification type (NT) string 
\\
{\tt\strut notifySubType} & pointer to string containing NT subtype 
\\
{\tt\strut usn} & pointer to string containing USN header 
\\
{\tt\strut location} & pointer to string containing Location header 
\\
{\tt\strut timeout} & pointer to max-age header value 
\end{tabular}}
\cxxReturn{
\begin{tabular}[t]{lp{0.5\textwidth}}
{\tt\strut error} & code
\end{tabular}}
\begin{cxxdoc}
This routine sends ssdp alive and bye-bye notifications on the mulicast
address for devices and servces


\end{cxxdoc}
\end{cxxfunction}
\begin{cxxfunction}
{SSDP\_INT32}
        {SSDP\_SendResponse}
        {(\ SSDPServerContext*\ ctx,\ \ SSDPPendingResponse*\ response)}
        {Deliver a responce to SSDP discovery request.}
        {2.1.6}
\cxxParameter{
\begin{tabular}[t]{lp{0.5\textwidth}}
{\tt\strut ctx} & pointer to SSDP context 
\\
{\tt\strut response} & buffer holding response information 
\end{tabular}}
\cxxReturn{
\begin{tabular}[t]{lp{0.5\textwidth}}
{\tt\strut error} & code
\end{tabular}}
\begin{cxxdoc}
Deliver a responce to SSDP discovery request.


\end{cxxdoc}
\end{cxxfunction}
\begin{cxxfunction}
{SSDP\_INT32}
        {\_SSDP\_ProcessOneRequest}
        {(\ SSDPServerContext*\ ctx\ )}
        {Process an incoming SSDP discovery request. }
        {2.1.7}
\cxxParameter{
\begin{tabular}[t]{lp{0.5\textwidth}}
{\tt\strut ctx} & pointer to SSDP context 
\end{tabular}}
\cxxReturn{
\begin{tabular}[t]{lp{0.5\textwidth}}
{\tt\strut error} & code
\end{tabular}}
\begin{cxxdoc}
Processes a SSDP discovery request available during a period given by
timeoutMsec milli seconds.


\end{cxxdoc}
\end{cxxfunction}
\begin{cxxfunction}
{int}
        {SSDP\_ParseRequest}
        {(\ SSDPServerContext*\ ctx,\ \ SSDPServerRequest*\ ssdpRequest\ )}
        {Extract SSDP request. }
        {2.1.8}
\cxxParameter{
\begin{tabular}[t]{lp{0.5\textwidth}}
{\tt\strut ctx} & pointer to SSDP context 
\\
{\tt\strut ssdpRequest} & address of the SSDPServerEvent structure to fill up 
\end{tabular}}
\cxxReturn{
\begin{tabular}[t]{lp{0.5\textwidth}}
{\tt\strut error} & code
\end{tabular}}
\begin{cxxdoc}
Retrieves messages from the multicast address, if ssdp request is
detected a buffer holding request information is populated


\end{cxxdoc}
\end{cxxfunction}
\begin{cxxfunction}
{int}
        {SSDP\_McastRead}
        {(\ void*\ cookie,\ \ SSDP\_UINT8*\ buffer,\ \ SSDP\_INT32\ min,\ \ SSDP\_INT32\ max)}
        {Reads all messages posted to the multicast address}
        {2.1.9}
\cxxParameter{
\begin{tabular}[t]{lp{0.5\textwidth}}
{\tt\strut cookie} & internal cookie 
\\
{\tt\strut buffer} & pointer to buffer contaning request message 
\\
{\tt\strut min} & \\
{\tt\strut max} & max size to be read 
\end{tabular}}
\cxxReturn{
\begin{tabular}[t]{lp{0.5\textwidth}}
{\tt\strut error} & code
\end{tabular}}
\begin{cxxdoc}
Reads all messages posted to the multicast address


\end{cxxdoc}
\end{cxxfunction}
\begin{cxxfunction}
{int}
        {\_SSDP\_ReadMSearchHeader}
        {(\ void*\ request,\ \ HTTPSession*\ ptr,\ \ HTTPHeaderType\ type,\ \ const\ HTTP\_CHAR*\ name,\ \ const\ HTTP\_CHAR*\ value\ )}
        {Extracts MX and St headers from a SSDP request}
        {2.1.10}
\cxxParameter{
\begin{tabular}[t]{lp{0.5\textwidth}}
{\tt\strut request} & request buffer to be populated 
\\
{\tt\strut ptr} & current HTTP session 
\\
{\tt\strut type} & HTTP header type 
\\
{\tt\strut name} & holds the name of the header 
\\
{\tt\strut value} & holds the value of the header
\end{tabular}}
\cxxReturn{
\begin{tabular}[t]{lp{0.5\textwidth}}
{\tt\strut error} & code
\end{tabular}}
\begin{cxxdoc}
Extracts MX and St headers from a SSDP request


\end{cxxdoc}
\end{cxxfunction}
\begin{cxxfunction}
{int}
        {\_SSDP\_ReadNotifyHeader}
        {(\ void*\ request,\ \ HTTPSession*\ ptr,\ \ HTTPHeaderType\ type,\ \ const\ HTTP\_CHAR*\ name,\ \ const\ HTTP\_CHAR*\ value\ )}
        {Extracts MX and St headers from a SSDP request}
        {2.1.11}
\cxxParameter{
\begin{tabular}[t]{lp{0.5\textwidth}}
{\tt\strut request} & request buffer to be populated 
\\
{\tt\strut ptr} & current HTTP session 
\\
{\tt\strut type} & HTTP header type 
\\
{\tt\strut name} & holds the name of the header 
\\
{\tt\strut value} & holds the value of the header
\end{tabular}}
\cxxReturn{
\begin{tabular}[t]{lp{0.5\textwidth}}
{\tt\strut error} & code
\end{tabular}}
\begin{cxxdoc}
Extracts MX and St headers from a SSDP request


\end{cxxdoc}
\end{cxxfunction}
\begin{cxxfunction}
{SSDP\_INT32}
        {SSDP\_QueueSearchResponse}
        {(\ SSDPServerContext*\ ctx,\ \ SSDPSearch*\ search,\ \ const\ SSDP\_CHAR*\ targetLocation,\ \ const\ SSDP\_CHAR*\ targetURN,\ \ SSDP\_UINT32\ targetTimeoutSec)}
        {Queues a response to the response list based on its scheduled delivery time.}
        {2.1.12}
\cxxParameter{
\begin{tabular}[t]{lp{0.5\textwidth}}
{\tt\strut ctx} & pointer to SSDP context 
\\
{\tt\strut search} & pointer to the buffer containingthe request information 
\\
{\tt\strut targetLocation} & pointer to string containing Location header
\\
{\tt\strut targetURN} & pointer to string containing USN header 
\\
{\tt\strut targetTimeoutSec} & max-age header value 
\end{tabular}}
\cxxReturn{
\begin{tabular}[t]{lp{0.5\textwidth}}
{\tt\strut error} & code
\end{tabular}}
\begin{cxxdoc}
Queues a response to the response list based on its scheduled delivery time.
A random delivery time within targetTimeoutSec duration is calculated.
This response is positioned in the list according to its scheduled delivery time.


\end{cxxdoc}
\end{cxxfunction}
\begin{cxxfunction}
{SSDP\_INT32}
        {SSDP\_CheckPendingResponses}
        {(\ SSDPServerContext*\ ctx,\ \ SSDP\_UINT32\ currentTimeMsec)}
        {Delivers responses scheduled for delivery. }
        {2.1.13}
\cxxParameter{
\begin{tabular}[t]{lp{0.5\textwidth}}
{\tt\strut ctx} & pointer to SSDP context 
\\
{\tt\strut currentTimeMsec} & time against which the scheduledtime is checked 
\end{tabular}}
\cxxReturn{
\begin{tabular}[t]{lp{0.5\textwidth}}
{\tt\strut error} & code
\end{tabular}}
\begin{cxxdoc}
Scan the pending response list and deliver responses for which the
scheduled time count is less than supplied time count.


\end{cxxdoc}
\end{cxxfunction}
\begin{cxxfunction}
{void}
        {SSDP\_ProcessError}
        {(\ SSDP\_CHAR*\ errMsg)}
        {Process SSDP Errors}
        {2.1.14}
\cxxParameter{
\begin{tabular}[t]{lp{0.5\textwidth}}
{\tt\strut errMsg} & error message string 
\end{tabular}}
\cxxReturn{
\begin{tabular}[t]{lp{0.5\textwidth}}
{\tt\strut None} & \end{tabular}}
\begin{cxxdoc}
Process SSDP Errors


\end{cxxdoc}
\end{cxxfunction}
\begin{cxxfunction}
{SSDP\_UINT32}
        {SSDP\_RandMax}
        {(\ SSDP\_UINT32\ mxLimit)}
        {generates a random number between 0 and mxLimit}
        {2.1.15}
\cxxParameter{
\begin{tabular}[t]{lp{0.5\textwidth}}
{\tt\strut mxLimit} & upper limit of the random number 
\end{tabular}}
\cxxReturn{
\begin{tabular}[t]{lp{0.5\textwidth}}
{\tt\strut error} & code
\end{tabular}}
\begin{cxxdoc}
generates a random number between 0 and mxLimit


\end{cxxdoc}
\end{cxxfunction}
\begin{cxxfunction}
{int}
        {UPnP\_DeviceSSDPCallback}
        {(\ SSDPServerContext*\ ctx,\ \ SSDPServerRequest*\ serverRequest,\ \ void*\ cookie\ )}
        {SSDP callback for UPnP. }
        {2.1.16}
\cxxParameter{
\begin{tabular}[t]{lp{0.5\textwidth}}
{\tt\strut ctx} & the SSDP context 
\\
{\tt\strut serverRequest} & address of buffer holding ssdp requestinformation 
\\
{\tt\strut cookie} & cookie holds pointer to device run timeinformation 
\end{tabular}}
\cxxReturn{
\begin{tabular}[t]{lp{0.5\textwidth}}
{\tt\strut error} & code
\end{tabular}}
\begin{cxxdoc}
The callback function checks the type of SSDP request and creates a
corresponding response. This response is queued in a list with a scheduled
delivery time


\end{cxxdoc}
\end{cxxfunction}
\begin{cxxfunction}
{int}
        {UPnP\_DeviceSendAllAlive}
        {(\ UPnPDeviceRuntime*\ runtime\ )}
        {Sends alive advertisements for everything under all the root devices. }
        {2.1.17}
\cxxParameter{
\begin{tabular}[t]{lp{0.5\textwidth}}
{\tt\strut runtime} & address of devices current runtime state 
\end{tabular}}
\cxxReturn{
\begin{tabular}[t]{lp{0.5\textwidth}}
{\tt\strut error} & code
\end{tabular}}
\begin{cxxdoc}
Sends alive advertisements for each device and associated service for
all the root devices.


\end{cxxdoc}
\end{cxxfunction}
\begin{cxxfunction}
{int}
        {UPnP\_DeviceSendRootDeviceAlive}
        {(\ UPnPRootDevice*\ rootDevice,\ \ int\ deep)}
        {Sends alive advertisements for a root device.}
        {2.1.18}
\cxxParameter{
\begin{tabular}[t]{lp{0.5\textwidth}}
{\tt\strut rootDevice} & address of the root device 
\\
{\tt\strut deep} & if 1, send advertisements for all embedded dives and services\<br\>if 0, send advertisements for the device under the root and itsassociated services 
\end{tabular}}
\cxxReturn{
\begin{tabular}[t]{lp{0.5\textwidth}}
{\tt\strut error} & code
\end{tabular}}
\begin{cxxdoc}
Sends alive advertisements for a root device.


\end{cxxdoc}
\end{cxxfunction}
\begin{cxxfunction}
{int}
        {UPnP\_DeviceSendDeviceAlive}
        {(\ UPnPDevice*\ device)}
        {Sends alive notifications for a device}
        {2.1.19}
\cxxParameter{
\begin{tabular}[t]{lp{0.5\textwidth}}
{\tt\strut device} & pointer to the device 
\end{tabular}}
\cxxReturn{
\begin{tabular}[t]{lp{0.5\textwidth}}
{\tt\strut error} & code
\end{tabular}}
\begin{cxxdoc}
Sends alive notifications for a device


\end{cxxdoc}
\end{cxxfunction}
\begin{cxxfunction}
{int}
        {UPnP\_DeviceSendServiceAlive}
        {(\ UPnPService*\ service)}
        {Sends alive notifications for a service}
        {2.1.20}
\cxxParameter{
\begin{tabular}[t]{lp{0.5\textwidth}}
{\tt\strut service} & pointer to a service 
\end{tabular}}
\cxxReturn{
\begin{tabular}[t]{lp{0.5\textwidth}}
{\tt\strut error} & code
\end{tabular}}
\begin{cxxdoc}
Sends alive notifications for a service


\end{cxxdoc}
\end{cxxfunction}
\begin{cxxfunction}
{int}
        {\_UPnP\_DeviceSendDeviceAlive}
        {(\ UPnPDevice*\ device)}
        {Sends alive advertisements for a device.}
        {2.1.21}
\cxxParameter{
\begin{tabular}[t]{lp{0.5\textwidth}}
{\tt\strut device} & pointer to the device 
\end{tabular}}
\cxxReturn{
\begin{tabular}[t]{lp{0.5\textwidth}}
{\tt\strut error} & code
\end{tabular}}
\begin{cxxdoc}
Sends alive advertisements for a device.


\end{cxxdoc}
\end{cxxfunction}
\begin{cxxfunction}
{int}
        {\_UPnP\_DeviceSendServiceAlive}
        {(\ UPnPService*\ service)}
        {Sends alive advertisements for a service.}
        {2.1.22}
\cxxParameter{
\begin{tabular}[t]{lp{0.5\textwidth}}
{\tt\strut service} & pointer to the service 
\end{tabular}}
\cxxReturn{
\begin{tabular}[t]{lp{0.5\textwidth}}
{\tt\strut error} & code
\end{tabular}}
\begin{cxxdoc}
Sends alive advertisements for a service.


\end{cxxdoc}
\end{cxxfunction}
\begin{cxxfunction}
{int}
        {\_UPnP\_DeviceSendRootDeviceAlive}
        {(\ UPnPRootDevice*\ rootDevice,\ \ int\ deep)}
        {Sends alive advertisements for every thing under all root devices. }
        {2.1.23}
\cxxParameter{
\begin{tabular}[t]{lp{0.5\textwidth}}
{\tt\strut rootDevice} & pointer to the root device 
\\
{\tt\strut deep} & if 1, send advertisements for all embedded dives and services\<br\>if 0, send advertisements for the device under the root and itsassociated services 
\end{tabular}}
\cxxReturn{
\begin{tabular}[t]{lp{0.5\textwidth}}
{\tt\strut error} & code
\end{tabular}}
\begin{cxxdoc}
Sends alive advertisements for each device and associated service for all the
root devices.


\end{cxxdoc}
\end{cxxfunction}
\begin{cxxfunction}
{int}
        {UPnP\_DeviceSendAllByeBye}
        {(\ UPnPDeviceRuntime*\ runtime)}
        {Sends bye-bye advertisements for each device and associated service for all the root devices.}
        {2.1.24}
\cxxParameter{
\begin{tabular}[t]{lp{0.5\textwidth}}
{\tt\strut runtime} & address of devices current runtime state 
\end{tabular}}
\cxxReturn{
\begin{tabular}[t]{lp{0.5\textwidth}}
{\tt\strut error} & code
\end{tabular}}
\begin{cxxdoc}
Sends bye-bye advertisements for each device and associated service for all the
root devices.


\end{cxxdoc}
\end{cxxfunction}
\begin{cxxfunction}
{int}
        {UPnP\_DeviceSendRootDeviceByeBye}
        {(\ UPnPRootDevice*\ rootDevice,\ \ int\ deep)}
        {Sends bye-bye notifications for a root device}
        {2.1.25}
\cxxParameter{
\begin{tabular}[t]{lp{0.5\textwidth}}
{\tt\strut rootDevice} & address of the root device 
\\
{\tt\strut deep} & if 1, send advertisements for all embedded dives and services\<br\>if 0, send advertisements for the device under the root and itsassociated services 
\end{tabular}}
\cxxReturn{
\begin{tabular}[t]{lp{0.5\textwidth}}
{\tt\strut error} & code
\end{tabular}}
\begin{cxxdoc}
Sends bye-bye notifications for a root device


\end{cxxdoc}
\end{cxxfunction}
\begin{cxxfunction}
{int}
        {UPnP\_DeviceSendDeviceByeBye}
        {(\ UPnPDevice*\ device)}
        {Sends bye-bye notifications for the device}
        {2.1.26}
\cxxParameter{
\begin{tabular}[t]{lp{0.5\textwidth}}
{\tt\strut device} & pointer to the device 
\end{tabular}}
\cxxReturn{
\begin{tabular}[t]{lp{0.5\textwidth}}
{\tt\strut error} & code
\end{tabular}}
\begin{cxxdoc}
Sends bye-bye notifications for the device


\end{cxxdoc}
\end{cxxfunction}
\begin{cxxfunction}
{int}
        {UPnP\_DeviceSendServiceByeBye}
        {(\ UPnPService*\ service)}
        {Sends bye-bye notifications for the service}
        {2.1.27}
\cxxParameter{
\begin{tabular}[t]{lp{0.5\textwidth}}
{\tt\strut service} & pointer to the service 
\end{tabular}}
\cxxReturn{
\begin{tabular}[t]{lp{0.5\textwidth}}
{\tt\strut error} & code
\end{tabular}}
\begin{cxxdoc}
Sends bye-bye notifications for the service


\end{cxxdoc}
\end{cxxfunction}
\begin{cxxfunction}
{int}
        {\_UPnP\_DeviceSendRootDeviceByeBye}
        {(\ UPnPRootDevice*\ rootDevice,\ \ int\ deep)}
        {Sends bye-bye notifications for root device.}
        {2.1.28}
\cxxParameter{
\begin{tabular}[t]{lp{0.5\textwidth}}
{\tt\strut rootDevice} & address of the root device 
\\
{\tt\strut deep} & if 1, send advertisements for all embedded devicesand services.\<br\>if 0, send advertisements for thedevice under the root and its associated services 
\end{tabular}}
\cxxReturn{
\begin{tabular}[t]{lp{0.5\textwidth}}
{\tt\strut error} & code
\end{tabular}}
\begin{cxxdoc}
Sends bye-bye notifications for root device.


\end{cxxdoc}
\end{cxxfunction}
\begin{cxxfunction}
{int}
        {\_UPnP\_DeviceSendDeviceByeBye}
        {(\ UPnPDevice*\ device)}
        {Sends bye-bye notifications for the device}
        {2.1.29}
\cxxParameter{
\begin{tabular}[t]{lp{0.5\textwidth}}
{\tt\strut device} & address of the device 
\end{tabular}}
\cxxReturn{
\begin{tabular}[t]{lp{0.5\textwidth}}
{\tt\strut error} & code
\end{tabular}}
\begin{cxxdoc}
Sends bye-bye notifications for the device


\end{cxxdoc}
\end{cxxfunction}
\begin{cxxfunction}
{int}
        {\_UPnP\_DeviceSendServiceByeBye}
        {(\ UPnPService*\ service)}
        {Sends bye-bye notifications a service.}
        {2.1.30}
\cxxParameter{
\begin{tabular}[t]{lp{0.5\textwidth}}
{\tt\strut service} & address of the service 
\end{tabular}}
\cxxReturn{
\begin{tabular}[t]{lp{0.5\textwidth}}
{\tt\strut error} & code
\end{tabular}}
\begin{cxxdoc}
Sends bye-bye notifications a service.


\end{cxxdoc}
\end{cxxfunction}
\end{cxxentry}
\begin{cxxentry}
{}
        {Description}
        {}
        {}
        {2.2}
\begin{cxxnames}
\cxxitem{UPnPRootDevice*}
        {UPnP\_DeviceDescribeRootDevice}
        {(\ IXML\_Document*\ doc,\ \ int\ maxDepth)}
        {Get the root device elements needed by SSDP, GENA and SOAP.}
        {2.2.1}
\cxxitem{UPnPDevice*}
        {UPnP\_DeviceDescribeDevice}
        {(\ IXML\_Element*\ deviceElement,\ \ UPnPRootDevice*\ rootDevice,\ \ int\ maxDepth)}
        {Extracts the device information needed by SSDP, GENA and SOAP.}
        {2.2.2}
\cxxitem{void}
        {\_UPnP\_DeviceDescribeDevice}
        {(\ UPnPDevice*\ device,\ \ IXML\_Element*\ deviceElement,\ \ UPnPRootDevice*\ rootDevice,\ \ int\ maxDepth)}
        {Extracts device information.}
        {2.2.3}
\cxxitem{UPnPService*}
        {UPnP\_DeviceDescribeService}
        {(\ IXML\_Element*\ serviceElement,\ \ UPnPDevice*\ device)}
        {Extracts the service information.}
        {2.2.4}
\cxxitem{void}
        {\_UPnP\_DeviceDescribeService}
        {(\ UPnPService*\ serviceNode,\ \ IXML\_Element*\ serviceElement,\ \ UPnPDevice*\ device)}
        {Extracts service information.}
        {2.2.5}
\cxxitem{void}
        {UPnP\_DeviceFreeRootDevice}
        {(\ UPnPRootDevice*\ rootDevice)}
        {Frees the resources used by a UPnPRootDevice type root device}
        {2.2.6}
\cxxitem{void}
        {UPnP\_DeviceFreeDevice}
        {(\ UPnPDevice*\ device)}
        {Frees the resources used by the device}
        {2.2.7}
\cxxitem{void}
        {UPnP\_DeviceFreeService}
        {(\ UPnPService*\ service)}
        {Frees the resources used by the device}
        {2.2.8}
\end{cxxnames}
\begin{cxxdoc}
These functions are \<b\>not\</b\> at the API level and should not be called from
code outside the UPnP library. 
\end{cxxdoc}
\begin{cxxfunction}
{UPnPRootDevice*}
        {UPnP\_DeviceDescribeRootDevice}
        {(\ IXML\_Document*\ doc,\ \ int\ maxDepth)}
        {Get the root device elements needed by SSDP, GENA and SOAP.}
        {2.2.1}
\cxxParameter{
\begin{tabular}[t]{lp{0.5\textwidth}}
{\tt\strut doc} & pointer to device document xml page 
\\
{\tt\strut maxDepth} & indicates the depth level this funtionwill search the dom tree for devicesembedded within this device.\<br\>If maxDepth= x, where x \> 0, then UPnPDevicebuffer will search for embeddeddevices and related services that are xlevel deep in the XML DOM tree.\<br\>IfmaxDepth = 0 then only the device and itsrelated service information are described 
\end{tabular}}
\cxxReturn{
\begin{tabular}[t]{lp{0.5\textwidth}}
{\tt\strut Address} & of UPnPRootDevice buffer containing root device information
\end{tabular}}
\begin{cxxdoc}
Get the root device elements needed by SSDP, GENA and SOAP.


\end{cxxdoc}
\end{cxxfunction}
\begin{cxxfunction}
{UPnPDevice*}
        {UPnP\_DeviceDescribeDevice}
        {(\ IXML\_Element*\ deviceElement,\ \ UPnPRootDevice*\ rootDevice,\ \ int\ maxDepth)}
        {Extracts the device information needed by SSDP, GENA and SOAP.}
        {2.2.2}
\cxxParameter{
\begin{tabular}[t]{lp{0.5\textwidth}}
{\tt\strut deviceElement} & location of device information insidexml dom tree 
\\
{\tt\strut rootDevice} & pointer to the device's root device 
\\
{\tt\strut maxDepth} & indicates the depth level this funtionwill search the dom tree for devicesembedded within this device.\<br\>If maxDepth= x, where x \> 0, then UPnPDevicebuffer will search for embeddeddevices and related services that are xlevel deep in the XML DOM tree.\<br\>IfmaxDepth = 0 then only the device and itsrelated service information are described 
\end{tabular}}
\cxxReturn{
\begin{tabular}[t]{lp{0.5\textwidth}}
{\tt\strut Address} & of UPnPDevice buffer having device information
\end{tabular}}
\begin{cxxdoc}
Extracts the device information needed by SSDP, GENA and SOAP.


\end{cxxdoc}
\end{cxxfunction}
\begin{cxxfunction}
{void}
        {\_UPnP\_DeviceDescribeDevice}
        {(\ UPnPDevice*\ device,\ \ IXML\_Element*\ deviceElement,\ \ UPnPRootDevice*\ rootDevice,\ \ int\ maxDepth)}
        {Extracts device information.}
        {2.2.3}
\cxxParameter{
\begin{tabular}[t]{lp{0.5\textwidth}}
{\tt\strut device} & pointer to UPnPDevice buffer to storeextracted information in 
\\
{\tt\strut deviceElement} & location of device information insidexml dom tree 
\\
{\tt\strut rootDevice} & pointer to the device's root device 
\\
{\tt\strut maxDepth} & indicates the depth level this funtionwill search the dom tree for devicesembedded within this device 
\end{tabular}}
\cxxReturn{
\begin{tabular}[t]{lp{0.5\textwidth}}
{\tt\strut None} & \end{tabular}}
\begin{cxxdoc}
Extracts device information.  Gets the required information from a XML
document searching maxDepth deep for any embedded devices and related
services under this device


\end{cxxdoc}
\end{cxxfunction}
\begin{cxxfunction}
{UPnPService*}
        {UPnP\_DeviceDescribeService}
        {(\ IXML\_Element*\ serviceElement,\ \ UPnPDevice*\ device)}
        {Extracts the service information.}
        {2.2.4}
\cxxParameter{
\begin{tabular}[t]{lp{0.5\textwidth}}
{\tt\strut serviceElement} & location of service information insidexml dom tree 
\\
{\tt\strut device} & address of device this service belongsto 
\end{tabular}}
\cxxReturn{
\begin{tabular}[t]{lp{0.5\textwidth}}
{\tt\strut Address} & of UPnPService buffer containing service information
\end{tabular}}
\begin{cxxdoc}
Extracts the service information.


\end{cxxdoc}
\end{cxxfunction}
\begin{cxxfunction}
{void}
        {\_UPnP\_DeviceDescribeService}
        {(\ UPnPService*\ serviceNode,\ \ IXML\_Element*\ serviceElement,\ \ UPnPDevice*\ device)}
        {Extracts service information.}
        {2.2.5}
\cxxParameter{
\begin{tabular}[t]{lp{0.5\textwidth}}
{\tt\strut serviceNode} & pointer to UPnPService buffer to storeextracted information in 
\\
{\tt\strut serviceElement} & location of service information insidexml dom tree 
\\
{\tt\strut device} & address of device this service belongsto 
\end{tabular}}
\cxxReturn{
\begin{tabular}[t]{lp{0.5\textwidth}}
{\tt\strut None} & \end{tabular}}
\begin{cxxdoc}
Extracts service information. Gets the service related information from a XML
document


\end{cxxdoc}
\end{cxxfunction}
\begin{cxxfunction}
{void}
        {UPnP\_DeviceFreeRootDevice}
        {(\ UPnPRootDevice*\ rootDevice)}
        {Frees the resources used by a UPnPRootDevice type root device}
        {2.2.6}
\cxxParameter{
\begin{tabular}[t]{lp{0.5\textwidth}}
{\tt\strut rootDevice} & pointer to root device whose resourcesare to be freed 
\end{tabular}}
\cxxReturn{
\begin{tabular}[t]{lp{0.5\textwidth}}
{\tt\strut None} & \end{tabular}}
\begin{cxxdoc}
Frees the resources used by a UPnPRootDevice type root device


\end{cxxdoc}
\end{cxxfunction}
\begin{cxxfunction}
{void}
        {UPnP\_DeviceFreeDevice}
        {(\ UPnPDevice*\ device)}
        {Frees the resources used by the device}
        {2.2.7}
\cxxParameter{
\begin{tabular}[t]{lp{0.5\textwidth}}
{\tt\strut device} & pointer to device whose resources are to be freed 
\end{tabular}}
\cxxReturn{
\begin{tabular}[t]{lp{0.5\textwidth}}
{\tt\strut None} & \end{tabular}}
\begin{cxxdoc}
Frees the resources used by the device


\end{cxxdoc}
\end{cxxfunction}
\begin{cxxfunction}
{void}
        {UPnP\_DeviceFreeService}
        {(\ UPnPService*\ service)}
        {Frees the resources used by the device}
        {2.2.8}
\cxxParameter{
\begin{tabular}[t]{lp{0.5\textwidth}}
{\tt\strut service} & pointer to service whose resources are to be freed 
\end{tabular}}
\cxxReturn{
\begin{tabular}[t]{lp{0.5\textwidth}}
{\tt\strut None} & \end{tabular}}
\begin{cxxdoc}
Frees the resources used by the device


\end{cxxdoc}
\end{cxxfunction}
\end{cxxentry}
\begin{cxxentry}
{}
        {Control/SOAP}
        {}
        {}
        {2.3}
\begin{cxxnames}
\cxxitem{int}
        {UPnP\_DeviceControlBindService}
        {(\ UPnPDeviceRuntime*\ deviceRuntime,\ \ UPnPService*\ service)}
        {Initialize a service for control.}
        {2.3.1}
\cxxitem{int}
        {UPnP\_DeviceControlUnBindService}
        {(\ UPnPDeviceRuntime*\ deviceRuntime,\ \ UPnPService*\ service)}
        {Stop control on a service.}
        {2.3.2}
\cxxitem{int}
        {UPnP\_DeviceControlSOAPCallback}
        {(\ SOAPServerContext*\ ctx,\ \ SOAPRequest*\ request,\ \ void*\ cookie)}
        {SOAP callback for UPnP.}
        {2.3.3}
\cxxitem{char*}
        {\_getControlURL}
        {(\ IXML\_Document*\ descDoc,\ \ IXML\_Element*\ serviceElement)}
        {Calculates server-relative control URL for a service.}
        {2.3.4}
\end{cxxnames}
\begin{cxxdoc}
These functions are \<b\>not\</b\> at the API level and should not be called from
code outside the UPnP library. 
\end{cxxdoc}
\begin{cxxfunction}
{int}
        {UPnP\_DeviceControlBindService}
        {(\ UPnPDeviceRuntime*\ deviceRuntime,\ \ UPnPService*\ service)}
        {Initialize a service for control.}
        {2.3.1}
\cxxParameter{
\begin{tabular}[t]{lp{0.5\textwidth}}
{\tt\strut deviceRuntime} & the device runtime that owns theservice. 
\\
{\tt\strut service} & the service to bind 
\end{tabular}}
\cxxReturn{
\begin{tabular}[t]{lp{0.5\textwidth}}
{\tt\strut error} & code
\end{tabular}}
\begin{cxxdoc}
Initialize a service for control. Binds a service control URL to the
HTTP server to enable SOAP action processing for that service.


\end{cxxdoc}
\end{cxxfunction}
\begin{cxxfunction}
{int}
        {UPnP\_DeviceControlUnBindService}
        {(\ UPnPDeviceRuntime*\ deviceRuntime,\ \ UPnPService*\ service)}
        {Stop control on a service.}
        {2.3.2}
\cxxParameter{
\begin{tabular}[t]{lp{0.5\textwidth}}
{\tt\strut deviceRuntime} & the device runtime that owns theservice. 
\\
{\tt\strut service} & the service to unbind 
\end{tabular}}
\cxxReturn{
\begin{tabular}[t]{lp{0.5\textwidth}}
{\tt\strut error} & code
\end{tabular}}
\begin{cxxdoc}
Stop control on a service. Removes the HTTP server binding for this
service's control URL.


\end{cxxdoc}
\end{cxxfunction}
\begin{cxxfunction}
{int}
        {UPnP\_DeviceControlSOAPCallback}
        {(\ SOAPServerContext*\ ctx,\ \ SOAPRequest*\ request,\ \ void*\ cookie)}
        {SOAP callback for UPnP.}
        {2.3.3}
\cxxParameter{
\begin{tabular}[t]{lp{0.5\textwidth}}
{\tt\strut ctx} & the SOAP context 
\\
{\tt\strut request} & the action request 
\\
{\tt\strut cookie} & the UPnPService being controlled 
\end{tabular}}
\cxxReturn{
\begin{tabular}[t]{lp{0.5\textwidth}}
{\tt\strut error} & code; tells SOAP whether the action was successful orgenerated a fault
\end{tabular}}
\begin{cxxdoc}
SOAP callback for UPnP.  Handles a single incoming SOAP control request by
parsing the action document and calling the device callback.


\end{cxxdoc}
\end{cxxfunction}
\begin{cxxfunction}
{char*}
        {\_getControlURL}
        {(\ IXML\_Document*\ descDoc,\ \ IXML\_Element*\ serviceElement)}
        {Calculates server-relative control URL for a service.}
        {2.3.4}
\cxxParameter{
\begin{tabular}[t]{lp{0.5\textwidth}}
{\tt\strut descDoc} & the service's device description document 
\\
{\tt\strut serviceElement} & the service element inside descDoc 
\end{tabular}}
\cxxReturn{
\begin{tabular}[t]{lp{0.5\textwidth}}
{\tt\strut error} & code
\end{tabular}}
\begin{cxxdoc}
Calculates server-relative control URL for a service.  For example, if the
service's absolute control URL is "http://192.168.1.110:5423/myService.0001",
this function will allocate and return the string: "/myService.0001".


\end{cxxdoc}
\end{cxxfunction}
\end{cxxentry}
\begin{cxxentry}
{}
        {Eventing/GENA}
        {}
        {}
        {2.4}
\begin{cxxdoc}
These functions are \<b\>not\</b\> at the API level and should not be called from
code outside the UPnP library. 
\end{cxxdoc}
\end{cxxentry}
\end{cxxentry}
\begin{cxxentry}
{}
        {Examples }
        {}
        {}
        {3}
\begin{cxxnames}
\cxxitem{}
        {UPnP Device Initialization Example}
        {}
        {Setting up a UPnP Device }
        {3.1}
\end{cxxnames}
\begin{cxxentry}
{}
        {UPnP Device Initialization Example}
        {}
        {Setting up a UPnP Device }
        {3.1}
\begin{cxxdoc}

This code demonstrates in brief the necessary steps to set up a UPnP device
for discovery, description, control, and eventing.

\<pre\>
int main (void)
{
int result;
IXML\_Document *xmlDevice;
UPnPRuntime rt;
UPnPRootDeviceHandle rootDevice;


// UPnP maintains a runtime structure; The first step is to
// initialize UPnPRuntime struct. UPnP\_RuntimeInit takes a
// pointer to an uninitialized UPnPRuntime struct and other
// necessary necessary data to initialize and populate upnp
// the engine.
result = UPnP\_RuntimeInit (
\&rt,
0,                  // serverAddr: IP\_ANY\_ADDR
0,                  // serverPort: any port
RTP\_NET\_TYPE\_IPV4,  // ipv4
"c:\\www-root\\",   // web server root dir
10,                 // maxConnections
5                   // maxHelperThreads
);

if (result \< 0)
{
return (-1);
}

// Next, we need a UPnPDeviceRuntime; UPnP\_DeviceInit takes
//  a pointer to an uninitialized UPnPDeviceRuntime struct
//  and does all necessary initialization.

result = UPnP\_DeviceInit (
\&deviceRuntime,
\&rt
);

if (result \< 0)
{
return (-1);
}

// Load the root device description page into memory.
xmlDevice = ixmlLoadDocument("c:\\www-root\\device.xml");
if (!xmlDevice)
{
return (-1);
}


result = UPnP\_RegisterRootDevice (
\&deviceRuntime,
"device.xml",
xmlDevice,
1,                  // auto address resolution
testDeviceCallback,
0,                  // userData for callback
\&rootDevice,
1                   // advertise
);

if (result \< 0)
{
return (-1);
}

UPnP\_DeviceAdvertise(rootDevice, ANNOUNCE\_FREQUENCY\_SEC,
REMOTE\_CACHE\_TIMEOUT\_SEC);

// start the UPnP daemon thread
UPnP\_StartDaemon(\&rt);

// for polled mode, use this

//while (1)
//{
//UPnP\_ProcessState (\&rt,1000);
//printf(".");
//}

}

\</pre\>

\end{cxxdoc}
\end{cxxentry}
\end{cxxentry}
\end{document}
